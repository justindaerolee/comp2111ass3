\documentclass[headings=small,a4paper,12pt]{scrartcl}
\usepackage{nicefrac}
\usepackage{2111defs2}
%
%\newcommand{\sorted}[3]{\mathit{s'ed}(#1[#2..#3])}
\newcommand{\sort}[3]{\mathit{sort}(#1[#2..#3])}
\newcommand{\pre}{\mathit{pre}}
\newcommand{\post}{\mathit{post}}
\newcommand{\bfs}{\textsc{breath first search}\xspace}
\newcommand{\srh}{\textsc{search}\xspace}
\newcommand{\qu}{\textsc{queue}\xspace}
%
\title{Derivation of $\bfs$\\
\large COMP2111 assignment 3}
%
\author{Dae Ro Lee z5060887 and Wing Feng z5091907}
\allowdisplaybreaks
%
\begin{document}
\maketitle
%
\section{Introduction}
\label{sec:introduction}
The derivation of $\bfs$ on a tree using a bounded $\qu$. 
%
\section{The Derivation}
\label{sec:derivation}
\begin{align*}
  &\PROC~\srh(\VALUE~t, \VALUE~N, \VALUE~k, \RESULT~v, \RESULT~f)\cdot{}\\
   &\qquad  \nt{t,N,k,v,f:\left[
    \begin{array}{l}
     \forall x\in V_t(x\in \Gamma^{*}_{t}(r_t)\And x \notin\Gamma^{+}_{t}(x))\And\max_{i\in\mathbb{N}}|\Gamma_t^{i}(r_t)\cup\Gamma_{t}^{i+1}(r_t)|\leq N,\\
     (f\And\exists w\in V_{t_0}(k_t_0(w)=k_0\And\lambda_{t_0}(w)=v))\Or\\
     (\Not f\And\forall w\in V_{t_0}(k_{t_0}(w)\neq~k_0))
    \end{array}
  \right]}{(1)}\\
%
  \lrefstep{(1)}
  {\textbf{c-frame}}
  {v,f:\left[
    \begin{array}{l}
     \forall x\in V_t(x\in \Gamma^{*}_{t}(r_t)\And x \notin\Gamma^{+}_{t}(x))\And\max_{i\in\mathbb{N}}|\Gamma_t^{i}(r_t)\cup\Gamma_{t}^{i+1}(r_t)|\leq N,\\
     (f\And\exists w\in V_{t}(k_t(w)=k\And\lambda_t(w)=v))\Or\\
     (\Not f\And\forall w\in V_t(k_t(w)\neq~k))
    \end{array}
  \right]}\\
  \refstep{\textbf{introduce local variable}}
  {\VAR~q\cdot{}\nt{q,v,f:\left[
    \begin{array}{l}
     \forall x\in V_t(x\in \Gamma^{*}_{t}(r_t)\And x \notin\Gamma^{+}_{t}(x))\And\max_{i\in\mathbb{N}}|\Gamma_t^{i}(r_t)\cup\Gamma_{t}^{i+1}(r_t)|\leq N,\\
     (f\And\exists w\in V_{t}(k_t(w)=k\And\lambda_t(w)=v))\Or\\
     (\Not f\And\forall w\in~V_t(k_t(w)\neq~k))
    \end{array}
  \right]}{(2)}}\\
\end{align*}
%---the following are format for invariant, can be moved to correct position later
To help define the loop invariant, we define expression $path(t,r_t,x)$ as:\\
\begin{align*}
{\exists n, x \in \Gamma^{n}_{t}(r_t) }\\
\end{align*}
which means node x is reachable in tree t \\
\break
 We define the loop invariant for $\bfs$ as:\\
\begin{align*}
    {I:=\left[
      \begin{array}{l}
       % the visited here means the visited set which should be introduced before
       \Not f\left(
	   \begin{array}{l}
		  \forall x\in visited \cup q, path(t,r_t,x)\\
		  \And~\forall x \in visited, k_t(x)\neq k\\
		  \And~\Gamma_{t}(x) \subseteq visited \cup q
	    \end{array}
		\right )\\
      \Or (f\implies\exists x (k_t(x)=k \And \lambda_t(x)=v \And path(t,r_t,x)))
      \end{array}
    \right]}\\
\end{align*}
%---
Define the following queue operation:\\
\textbf{initialise:} initialise a queue that can hold up to \textit{N} elements to the empty queue vlaue.\\
\break
\textbf{enqueue:} adds an item to a queue if there's a space available \\
\begin{align*}
  &\PROC~enqueue(q, \VALUE~v,n)\cdot{}\\
    &\qquad n,q:[n<N \And q=q_0\And q \neq <>,q=q_0\cup\{v\}]\\
%
  \refstep{\textbf{}}
  {}
\end{align*}
%
\textbf{dequeue:} return the oldest item in the queue and remove it form the queue\\
\break
\textbf{isempty:} return whether a queue is empty\\
\break

\end{document}
