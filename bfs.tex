\documentclass[headings=small,a4paper,12pt]{scrartcl}
\usepackage{nicefrac}
\usepackage{2111defs2}
\usepackage{amsmath}
%
%\newcommand{\sorted}[3]{\mathit{s'ed}(#1[#2..#3])}
\newcommand{\sort}[3]{\mathit{sort}(#1[#2..#3])}
\newcommand{\pre}{\mathit{pre}}
\newcommand{\post}{\mathit{post}}
\newcommand{\bfs}{\textsc{breath first search}\xspace}
\newcommand{\srh}{\textsc{search}\xspace}
\newcommand{\qu}{\textsc{queue}\xspace}
\newcommand{\enq}{\textsc{enqueue}\xspace}
\newcommand{\deq}{\textsc{dequeue}\xspace}
\newcommand{\ini}{\textsc{initialise}\xspace}
%
\title{Derivation of $\bfs$\\
\large COMP2111 assignment 3}
%
\author{Dae Ro Lee z5060887 and Wing Feng z5091907}
\allowdisplaybreaks
%
\begin{document}
\maketitle
%
\section{Introduction}
\label{sec:introduction}
The derivation of $\bfs$ on a tree using a bounded $\qu$.\\ 
%
%---following are queue operation
Firstly we define a queue, and operations on abstract queues, which we eliminate later with data refinement.
\begin{align*}
    &Q= \mathbb{N}\times V^*_t
\end{align*}
Therefore the queue holds a natural number that describes the Maximum number of nodes it can hold and zero or more nodes.\\
We define the following queue operation:\\
\textbf{initialise:} initialise a queue that can hold up to \textit{N} elements to the empty queue value.
\begin{align*} 
  \ini\qquad~q :[\True, q = (N,\epsilon)] \isrefinedby q\Ass\ini(N)
\end{align*}
\textbf{\enq:} adds an item to a queue if there's a space available 
\begin{align*}
  \enq\qquad q:[q=(N,s)\And |s| < N, q=(N,sx)]\isrefinedby \enq(q,x)
\end{align*}
\textbf{\deq:} return the oldest item in the queue and remove it form the queue\\
\begin{align*}
  \deq\qquad q,y:[q = (N,xs), y = x \And q = q_0] \isrefinedby y \Ass \deq(q) 
\end{align*}
%
\section{The Derivation}
\label{sec:derivation}
%
\begin{align*}
  &\PROC~\srh(\VALUE~t, \VALUE~N, \VALUE~k, \RESULT~v, \RESULT~f)\cdot{}\\
   &\qquad  \nt{t,N,k,v,f:\left[
    \begin{array}{l}
     \forall x\in V_t(x\in \Gamma^{*}_{t}(r_t)\And x \notin\Gamma^{+}_{t}(x))\And\max_{i\in\mathbb{N}}|\Gamma_t^{i}(r_t)\cup\Gamma_{t}^{i+1}(r_t)|\leq N,\\
     (f\And\exists w\in V_{t_0}(k_{t_0}(w)=k_0\And\lambda_{t_0}(w)=v))\Or\\
     (\Not f\And\forall w\in V_{t_0}(k_{t_0}(w)\neq~k_0))
    \end{array}
  \right]}{(1)}\\
%
\\
  \lrefstep{(1)}
  {\textbf{c-frame}}
  {v,f:\left[
    \begin{array}{l}
     \forall x\in V_t(x\in \Gamma^{*}_{t}(r_t)\And x \notin\Gamma^{+}_{t}(x))\And\max_{i\in\mathbb{N}}|\Gamma_t^{i}(r_t)\cup\Gamma_{t}^{i+1}(r_t)|\leq N,\\
     (f\And\exists w\in V_{t}(k_t(w)=k\And\lambda_t(w)=v))\Or\\
     (\Not f\And\forall w\in V_t(k_t(w)\neq~k))
    \end{array}
  \right]}\\
  \refstep{\textbf{introduce local variable}}
  {\VAR~q, n\cdot{}\nt{q,n,v,f:\left[
    \begin{array}{l}
     \forall x\in V_t(x\in \Gamma^{*}_{t}(r_t)\And x \notin\Gamma^{+}_{t}(x))\And\max_{i\in\mathbb{N}}|\Gamma_t^{i}(r_t)\cup\Gamma_{t}^{i+1}(r_t)|\leq N,\\
     (f\And\exists w\in V_{t}(k_t(w)=k\And\lambda_t(w)=v))\Or\\
     (\Not f\And\forall w\in~V_t(k_t(w)\neq~k))
    \end{array}
  \right]}{(2)}}\\
\end{align*}
%---the following are format for invariant, can be moved to correct position later
\section{Invariant}
\label{sec:invariant}
\text We define the loop invariant for $\bfs$ as:\\
\begin{align*}
    &I = \left(
    \begin{array}{l}
         \exists j\in \mathbb{N}\cdot{}(\forall~i<j\cdot{}\forall~x \in \Gamma^{i}_t(r_t)\cdot{}K_t(x) \ne k)\\
         \And~0\leq~n\leq~N\And \exists~l\in[0,n-1] \cdot{}q = (N,w_0..w_l..w_{n-1})\\
         \And~w_0..w_l\subseteq\Gamma^j_t(r_t)\\
         \And~w_{l+1}..w_{n-1} \subseteq \Gamma^{j+1}_t(r_t)\And n = |w_0..w_{n-1}|\\
         \And (\forall y\in V_t(y\in \Gamma^{*}_{t}(r_t)\And y \notin\Gamma^{+}_{t}(y)))\And\max_{i\in\mathbb{N}}|\Gamma_t^{i}(r_t)\cup\Gamma_{t}^{i+1}(r_t)|\leq N\\
         \And ((f\And\exists~s\in \Gamma^{j}_{t}(r_t)\cdot{}k_t(s) = k\And \lambda_t(s) = v\And s\notin q)\\
         \Or (\Not~f \And \forall~p \in \Gamma^j_t(r_t) \And p\notin~q \And k_t(p) \ne k \And \Gamma_t(p) \subseteq q))
    \end{array}
    \right)\\
\end{align*}
$j$ is defined as a node's distance from the root node, or the level of that node in the tree, hence at root node, j = 0.\\
%
The first line of the invariant shows that all previous levels of the tree before the current level has been visited and the key was not found.\\
the second line and third line of the invariant show that the queue, $q$ is composed of at most only two levels of the tree at any given time, $j$ (being defined at the current level) and $j+1$.\\
The sixth line of the invariant says that if $f$ then for some node on the current level there exists the node we are looking for.\\
The seventh line says all current level nodes that are not in $q$, are not the nodes we are looking for and their children are a subset of q.\\
\\
Since $0\leq~n\leq~N$, and $n$ is described as the length of the queue.\\ $\max_{i\in\mathbb{N}}|\Gamma_t^{i}(r_t)\cup\Gamma_{t}^{i+1}(r_t)|\leq N$, of the precondition implies the second and third line of the invariant. Showing that $n$ the length of the queue cannot exceed the number of nodes of two consecutive levels of the tree, hence $n\leq~N$, $N$ representing the largest number of nodes of two levels of the tree.\\
The invariant also contains the fact that all nodes are connected as a tree data structure, identical to that in the precondition.\\
\\
$I\And(f\Or~n=0)$, When $f=\True$ the second last line of the invariant describes that some node,\\ $s\in\Gamma^j_t(r_t)\cdot{}k_t(s) = k\And \lambda_t(s) = v\And s\notin q$, which implies the post condition in the scenario that $f = \True$\\
In the case that $f=\False\And~n=0$ the first and last line of the invariant shows that all nodes previous to current level have been checked.\\ Since the queue is empty all nodes on the current level have also been checked implying that none of the nodes in the tree contain the key that we are looking for,\\ implying the post condition in the scenario that $n=0$.\\
Therefore $I\And(f\Or~n=0)$ implies the post condition\\
Since the invariant says that $n$ represents the size of q (the number of nodes) we will increment n whenever $\enq$ is done or decrement n whenever $\deq$ is done such that the invariant is true at the beginning and end of the while loop.
%---
\break
\section{The Derivation continued ...}
\label{sec:derivation2}
\begin{align*}
\lrefstep{(2)}
{\textbf{seq} no initial variables} 
{
  \nt{q,n,v,f:[\pre(2), I] }{(3)}\\
  &;\nt{q,n,v,f:[I, \post(2)] }{(4)}\\
 }\\
%
\lrefstep{(3)}
{\textbf{seq} no initial variables} 
{
  \nt{q,n,v,f:[\pre(3), \pre(3)\And q=(N,\epsilon)] }{(5)}\\
  &;\nt{q,n,v,f:[\pre(3)\And q=(N,\epsilon),I] }{(6)}\\
 }     
%
\end{align*}
%
First we introduce $q : Q$ which will allow us to implement $\bfs$.\\ 
We then derive code by initialising $f$ to $\False$ and $\enq$ the root node to establish our invariant initially.
%
\begin{align*}
\lrefstep{(5)}
{\textbf{i-loc}} 
{
  \VAR~q:Q \Ass initialise(N)
 }\\      
%
\lrefstep{(6)}
{\textbf{seq, ass, \enq, ass}} 
{
  f \Ass \False;\\
  &\enq(q,r_t);\\
  & n \Ass 1\\
}\\
%
\lrefstep{(4)}
{\textbf{s-post} justified above in Sect.~\ref{sec:invariant}}
{q,n,v,f:[I\And~g, I\And (f \Or~ n=0)]}\\
%
\refstep{\textbf{\WHILE}} 
{
  \WHILE~n \neq 0 \And \Not f ~\DO\\
      &\qquad \nt{q,n,v,f:[I\And~n\ne~0\And\Not~f, I]}{(7)}\\
  &\OD
 }\\  
%
 \lrefstep{(7)}
 {\textbf{i-loc}}
 {\VAR~tmp\cdot{}\nt{tmp,q,n,v,f:[I\And~n\ne~0\And\Not~f, I]}{(8)}}\\
%
 \lrefstep{(8)}
 {\textbf{seq} no initial variables}
 {\nt{tmp,q,n,v,f:[n\ne~0\And\Not~f\And~I\And~q = (N, xy),q = (N, y)\And I\subst{s}{tmp}\subst{(N,xy)}{(N,y)}\And\Not f]}{(a)}\\
 &;\nt{tmp,q,n,v,f:[q = (N, y)\And I\subst{s}{tmp}\subst{(N,xy)}{(N,y)}\And\Not f, I]}{(b)}}\\
%
 \lrefstep{(b)}
 {\textbf{if}}
 {\IF~k_t(tmp) = k~\THEN\\
 &\qquad~\nt{tmp,q,n,v,f:[q = (N, y)\And I\subst{s}{tmp}\subst{(N,xy)}{(N,y)}\And\Not f\And k_t(tmp) = k, I]}{(c)}\\
 &\ELSE\\
 &\qquad \nt{tmp,q,n,v,f:[q = (N, y)\And I\subst{s}{tmp}\subst{(N,xy)}{(N,y)}\And\Not f\And k_t(tmp) \ne k, I]}{(d)}\\
 &\FI}\\
%
\lrefstep{(a)}
{\textbf{\deq}}
{tmp \Ass \deq(q,n);\\
&n\Ass~n-1}\\
\end{align*}
%
To make the invariant $\True$ if $k_t(tmp) = k$, we have to assign $f \Ass \True$ because the sixth line of the invariant tell us that $f$ cannot be $\False$ when $k_t(tmp) = k$.\\
Now make the fifth line of the invariant true make $\lambda_t(tmp) = v$ using assignment such that the invariant is $True$.
%
\begin{align*}
\lrefstep{(c)}
{\textbf{seq, ass}}
{f\Ass \True;\\
&v = \lambda_t(tmp)}\\
%
\end{align*}
Because $tmp$ is not the node we are looking for, we now have to make sure the conditions of the invariant are met. The invariant tells us that $q$ contains up-to two levels of the tree at any given time. We also know that since $tmp$ has been $\deq$ from $q$, $tmp \notin q$, hence the end of the 5th line of the invariant tells us that the children of all the nodes that are on the current level but not in $q$ must be a subset of $q$, therefore all the children of $tmp$ have to be added to $q$ to meet the invariant.\\
We do this by looping through the set of successors and $\enq$ them to $q$.\\
%
\\
\\
\begin{align*} 
    \lrefstep{(d)}
    {\textbf{...}}
    {\VAR~m\Ass\Gamma_t(tmp);\\
    &\WHILE~m \ne \emptyset~\DO\\
    &\qquad\VAR~c:\in~m;\\
    &\qquad\enq(q,c);\\
    &\qquad n \Ass n+1;\\
    &\qquad m\Ass m\backslash\{c\};\\
    &\OD\\}
\end{align*}
%
\section{Coupling Invariant}
\label{sec:coupling_invariant}
We do the data refinement to take the abstract queue to a more concrete representation using circular buffer, that is an array \textit{r} of  \textit{N+1} cells and two counters \textit{h} and \textit{t} for which \textit{h} = 0 and \textit{t} = -1 when \textit{r} is empty, otherwise \textit{h mod (N+1)} and \textit{t mod (N+1)}  the head and tail indexs of \textit{r} which represents the head and tail of the queue.\\
%
Using the Reynolds method, the coupling invariant would be:\\
\begin{align*} 
  &C: Q(V_t) \times V^*_t \times \mathbb{N}\times \mathbb{Z}  &\to & \mathbb{B}\\
  &C( (N,\epsilon),r,h,t)  &=  &(t - h+1 = 0)\\
  &C(q=(N,sx),r,h,t) &= &(r[t\mod (N+1)]=x \And C(q,r,h,t))\\
  &C(q=(N,xs),r,h,t) &= &(x = r[h\mod (N+1)] \And C(q,r,h,t))
\end{align*}
For $\ini$ operations:\\
\begin{align*} 
  &\nt{q,r,h,t:[\True, q=(N,\epsilon) \And C((N,\epsilon),r,h,t)]}{(I_1)}\\\\
%
  \lrefstep{(I_1)}
  {\textbf{sc, c-frame...}} 
  {
     \nt{q:[\True,q=(N,\epsilon)] }{(I_2)}\\
     &;\nt{r,h,t:[q=(N,\epsilon), q=(N,\epsilon) \And C((N,\epsilon),r,h,t)]] }{(I_3)}
   }\\    
%   
  \lrefstep{(I_2)}
  {\textbf{\ini}} 
  {
     q:=(N,\epsilon)
   }\\   
%   
  \lrefstep{(I_3)}
  {\textbf{ass}} 
  {
     h := 0 ;~t := -1
   }\\    
\end{align*}
%
For $\enq$ operations:\\
\begin{align*}
  &\nt{q,r,h,t:[ C(q,r,h,t), q=(N, sx) \And C(q,r,h,t)]}{(E_1)}\\\\
%
  \lrefstep{(E_1)}
  {\textbf{sc, c-frame...}} 
  {
    \CON~Q;\\
     &\nt{q:[C(Q,r,h,t) \And q = Q = (N,s), q=(N,sx)\And C(Q,r,h,t)] }{(E_2)}\\
     &;\nt{r,t:[ q=(N,sx)\And C(Q,r,h,t),  q=(N,sx)\And C(q,r,h,t)] }{(E_3)}\\
   }\\    
%   
  \lrefstep{(E_2)}
  {\textbf{\enq}} 
  {
     \enq(q,x) 
   }\\   
%   
  \lrefstep{(E_3)}
  {\textbf{ass}} 
  {
     t = t+1 ;~r[t \mod (N+1)] = x
   }\\   
\end{align*} 
%
For $\deq$ operations:\\
\begin{align*} 
  &\nt{x,q,r,h,t:[q=(N,ab)\And C(q,r,h,t),x=a \And q=(N,b) \And C(q,r,h,t)]}{(D_1)}\\\\
% 
  \lrefstep{(D_1)}
  {\textbf{sc, c-frame,w-pre...}} 
  {
     \nt{q,x:[q=(N,ab)\And C(q,r,h,t), x=a \And q=(N,b) \And C((N,ab),r,h,t) }{(D_2)}\\
     &;\nt{r,h,x:[q=(N,b) \And C((N,ab),r,h,t),  x=a \And q=(N,b) \And C(q,r,h,t)] }{(D_3)}\\
   }\\    
%   
  \lrefstep{(D_2)}
  {\textbf{\enq}} 
  {
     x=\deq(q) 
   }\\   
%   
  \lrefstep{(D_3)}
  {\textbf{ass}} 
  {
    x = r[h \mod (N+1)];~h = h+1
   }\\   
\end{align*}
\break
\\
\\
\\
\\
%
\section{Code}
\label{sec:code}
Putting the code together we have
\begin{align}
    &\textcolor{red}{\VAR~q\Ass\langle\rangle}\\
    &\textcolor{blue}{\VAR~r:V^*_t; \VAR~h:\mathbb{N}\Ass 0; \VAR~t:\mathbb{Z}\Ass -1}\\
    &\VAR~n:\mathbb{N}\Ass 0\\
    &f\Ass \False\\
    &\textcolor{red}{\enq(q,r_t)}\\
    &\textcolor{blue}{t\Ass t+1;~r[t \mod (N+1)]\Ass r_t}\\
    &n\Ass 1\\
    &\WHILE~n \ne 0\And \Not f~\DO\\
    &\textcolor{red}{\qquad \VAR~tmp\Ass \deq(q,n)}\\
    &\qquad\textcolor{blue}{tmp\Ass r[h \mod (N+1)];~h\Ass h+1}\\
    &\qquad n\Ass n-1\\
    &\qquad \IF~k_t(tmp) = k~\THEN\\
    &\qquad\qquad f\Ass \True\\
    &\qquad\qquad v\Ass \lambda_t(tmp)\\
    &\qquad \ELSE\\
    &\qquad\qquad \VAR~m\Ass \Gamma_t(tmp)\\
    &\qquad\qquad\WHILE~m \ne \emptyset~\DO\\
    &\qquad\qquad\qquad\VAR~c:\in~m\\
    &\textcolor{red}{\qquad\qquad\qquad\enq(q,c)}\\
    &\qquad\qquad\qquad\textcolor{blue}{t\Ass t+1;~r[t \mod (N+1)]\Ass c}\\
    &\qquad\qquad\qquad n \Ass n+1\\
    &\qquad\qquad\qquad m\Ass m\backslash\{c\}\\
    &\qquad\qquad\OD\\
    &\qquad\FI\\
    &\OD
\end{align}
Text in \textcolor{red}{red} are abstract queue operations. We complete the data refinement by replacing \textcolor{red}{red} lines with \textcolor{blue}{blue} lines\\
\end{document}